\documentclass[12pt]{article}

\title{SQL Tutorial}
\author{I. Rifa}

\begin{document}

\maketitle

\section{tables}
In order to create a table we use the followed command:
CREATE TABLE [name] (
id INTEGER PRIMARY KEY AUTOINCREMENT,
origin TEXT NOT NULL,
destination TEXT NOT NULL,
duration INTEGER NOT NULL
);
After the words, id, origin, destination, duration, we set the format of this
column. INTEGER, TEXT, etc. If we say NOT NULL, we say that this column must
have an entry.

\section{contraints}
\begin{itemize}
    \item CHECK : mutable values
    \item DEFAULT : immutable values
    \item PRIMARY KEY :
    \item UNIQUE :
\end{itemize}

\section{insert}
In order to inster data inside our created table:
INSERT INTO [name_table]
(origin, destination, duration)
VALUES ("New York", "London", 415);
\section{select}
If we wish to see the data inside of our table we use select command:
SELECT * FROM [name_table];
If we want to print just some columns of our table we specify:
SELECT origin, destination FROM [name_table];
If we want to print just a specific row of our table using the name of the
column:
SELECT * FROM [name_table] WHERE id = 3;
SELECT * FROM [name_table] WHERE origin = "New York";
SELECT * FROM [name_table] WHERE origin LIKE "%a%";
SELECT * FROM [name_table] WHERE origin IN ("New York", "Lima");
SELECT * FROM [name_table] WHERE duration > 500;
SELECT * FROM [name_table] WHERE duration > 500 AND destination = "Paris";

\section{UPDATE}
In order to update information already set in our table we use UPDATE command:
UPDATE [name_table]
	SET duration = 430
	WHERE origin = "New York"
	AND destination = "London";

\section{DELETE}
If we want to delete a row we use delete command:
DELETE FROM	[name_table] WHERE destination = "Tokyo";


\end{document}
