\documentclass[12pt]{article}

\title{Django tutorial}
\author{I. Rifarachi}

\begin{document}

\maketitle


\section{Working with QuerySet}

Django comes with a powerful database-abstraction API that lets you create,
retrieve, update and delete objects easily. The Django Object-relational Mapper
(ORM) is compatible with MySQL, PostgreSQL, SQLite, and Oracle. We can define
the database on settings.py.

\subsection{Creating objects}
user = User.objects.get(username='test')
The get() method allows you to retrieve a single object from the database. This
method expects one result that matches the query. If no results are returned by
the database, this method will raise DoesNotExist exception, and if the
database returns more than one result, it will raise MultipleObjectsReturned
exception.
post = Post.objects.create(title='hola', body='post body')
In order to make the data persistent we must use post.save() method. This
action performs and INSERT SQL statement behind the scenes. If we want to
create the object directly then we use the method create().

\subsection{Updating objects}
We change the field of the post and then we save it.
post.title = 'updating'
post.save()

\subsection{Retrieving objects}
Each Django model has at least one manager, and the default manager is called
"objects". You get a QuerySet object by using your models manager. To retrieve
all objects from a table, you just use the all() method on the default objects
manager, like this:
all_posts = Post.objects.all()

\subsection{filtering objects}
To filter a QuerySet you can use the filter() method of the manager. For
example, we can retrieve all posts published in the year 2015 using the
following QuerySet:
Post.objects.filter(publish_year=2015)
You can also filter by multiple fields. For example, we can retrieve all posts
published in 2015 by the author admin:
Post.objects.filter(publish_year=2015, author='admin')

\subsection{Using exclude()}
You can exclude certain results from your QuerySet using the exclude() method
of the manager. For example, we can retrieve all posts published in 2015 whose
title don't start by Why:
Post.objects.filter(publish_year=2015) \
			.exclude(title_startswith='Why')

\subsection{Using order_by()}
You can order results by different fields using the order_by() method of the
manager. For example, you can retrieve all objects ordered by their title:
Post.objects.order_by('title')

\subsection{Deleting objects}
If you want to delete an object, you can do it from the object instance:
post = Post.objects.get(id=1)
post.delete()

\begin{verbatim}
    git branch
\end{verbatim}
To create a new branch, we can use the command:
\begin{verbatim}
    git checkout -b [branch name]
\end{verbatim}
This command is used to switch betweent branches in Git.
The -b flag is used to create a new branch and switch to it at the same time.
To switch between branches, we can use the command:
\begin{verbatim}
    git checkout [branch name]
\end{verbatim}
To merge a branch into the master branch, we can use the command:
\begin{verbatim}
    git merge [branch name]
\end{verbatim}

\section{Update origin - branch}
git checkout [origin branch]
git pull
git checkout [branch]
git fetch origin
git merge origin/[origin_branch]
git commit - m "[message]"
git push origin [branch]

\end{document}
