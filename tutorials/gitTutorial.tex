\documentclass[12pt]{article}

\title{Git Tutorial}
\author{I. Rifarachi}

\begin{document}

\maketitle

\section{Git Tutorial}

\begin{itemize}
    \item git reset --hard [commit hash] : reset to a previous commit using the commit hash
    \item git reset --hard origin/master : reset to the last commit on the remote repository
    \item git commit -am "Message" : add and commit at the same time

\end{itemize}

\section{Branches}

In git, branches are used to develop features isolated from each other. The master branch is the "default" branch when you create a repository. Use other branches for development and merge them back to the master branch upon completion.
If we want to see in which branch we are, we can use the command:
\begin{verbatim}
    git branch
\end{verbatim}
To create a new branch, we can use the command:
\begin{verbatim}
    git checkout -b [branch name]
\end{verbatim}
This command is used to switch betweent branches in Git.
The -b flag is used to create a new branch and switch to it at the same time.
To switch between branches, we can use the command:
\begin{verbatim}
    git checkout [branch name]
\end{verbatim}
To merge a branch into the master branch, we can use the command:
\begin{verbatim}
    git merge [branch name]
\end{verbatim}

\section{Update origin - branch}
git checkout [origin branch]
git pull
git checkout [branch]
git fetch origin
git merge origin/[origin_branch]
git commit - m "[message]"
git push origin [branch]

\end{document}
