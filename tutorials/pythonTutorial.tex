\documentclass[12pt]{article}

\title{Python Tutorial}
\author{I. Rifa}

\begin{document}

\maketitle

\section{print}

In order to print a formatted string, you can use:
name = input("What is your name? ")
print(f"Hello {name}")

\section{conditionals}
\subsection{if/elif/else}
number = int(input("Number: "))
if number > 0:
    print("Number is positive")
elif number < 0:
    print("Number is negative")
else:
    print("Number is zero")

\section{sequences}
\begin{itemize}
    \item list : mutable values
    \item tuple : immutable values
    \item set : collection of unique values
    \item dictionary : collection of key-value pairs
\end{itemize}

\subsection{list}
names = ["Harry", "Ron", "Hermione"]
print(names)
print(names[0])
\subsection{tuple}
coordinates = (10.0, 20.0)
\subsection{set}
s = set()
s.add(1)
s.add(3)
s.add(5)
print(s)

\subsection{dictionary}
houses = {"Harry": "Gryffindor", "Draco": "Slytherin"}
houses["Hermione"] = "Gryffindor"
print(houses["Harry"])

\section{loops}
\subsection{for}
for i in range(6):
    print(i)

\section{functions}
def square(x):
    return x * x

for i in range(10):
    print(f"The square of {i} is {square(i)}")

\section{classes}
class Point():
    def __init__(self, x, y):
        self.x = x
        self.y = y
p = Point(3, 5)
print(p.x)
print(p.y)

\section{decorators}
def announce(f):
    def wrapper():
        print("About to run the function...")
        f()
        print("Done with the function.")
    return wrapper

@announce
def hello():
    print("Hello, world!")

\section{exceptions}
x = int(input("x: "))
y = int(input("y: "))
try:
    result = x / y
except ZeroDivisionError:
    print("Error: Cannot divide by 0.")
    sys.exit(1)

print(f"{x} / {y} = {result}")


\section{Django}
In order to create a new project, you can use:
django-admin startproject projectName

If we want to run the server, we can use:
python manage.py runserver

In order to create a new app, you can use:
python manage.py startapp appName

\end{document}
